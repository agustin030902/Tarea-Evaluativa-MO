\documentclass[11pt,a4paper]{article}

\usepackage[spanish]{babel}
\usepackage[utf8]{inputenc}
\usepackage[T1]{fontenc}
\usepackage{amsmath,amssymb,amsthm}
\usepackage{geometry}
\usepackage{hyperref}
\usepackage{graphicx}

\geometry{margin=2.5cm}

\title{Análisis Teórico y Numérico de la Optimización de\\
$f(x,y) = \dfrac{\arctan(x^2 + y^2)}{1 + x^2}$\\[0.5em]
\small Repositorio del proyecto: \url{https://github.com/agustin030902/Tare-Evaluativa-MO}}
\author{}
\date{ }

\theoremstyle{plain}
\newtheorem{proposicion}{Proposición}
\newtheorem{teorema}{Teorema}

\theoremstyle{definition}
\newtheorem{definicion}{Definición}

\begin{document}

\maketitle

\begin{abstract}
Este trabajo presenta un análisis teórico y numérico de la función
$f(x,y)=\dfrac{\arctan(x^2+y^2)}{1+x^2}$ desde la perspectiva de la Programación No Lineal.
Se demuestra la existencia y unicidad de un mínimo global, se estudian las propiedades
de diferenciabilidad y convexidad local, y se analizan varios métodos clásicos de
optimización, incluyendo Gradiente Descendente y el Método de Newton.
Los resultados numéricos confirman las predicciones teóricas sobre convergencia y
eficiencia de los algoritmos considerados.
\end{abstract}

\section{Introducción}

La optimización de funciones no lineales de varias variables es un problema central
en matemáticas aplicadas, con aplicaciones en ingeniería, ciencias de datos y física.
En este trabajo se estudia la función
\[
f(x,y)=\frac{\arctan(x^2+y^2)}{1+x^2},
\]
la cual presenta un comportamiento no trivial debido a su dependencia diferenciada
en las variables $x$ y $y$.

El objetivo principal es analizar sus propiedades analíticas relevantes para la
optimización y evaluar el desempeño de distintos algoritmos numéricos en la
búsqueda de su mínimo global.

\section{Propiedades básicas de la función}

La función $f$ está bien definida para todo $(x,y)\in\mathbb{R}^2$, ya que
$\arctan$ es continua en $\mathbb{R}$ y $1+x^2>0$ para todo $x$.
Además, al ser composición y cociente de funciones suaves, se tiene que
$f\in C^\infty(\mathbb{R}^2)$.

Notamos que
\[
\arctan(x^2+y^2)\geq 0,
\]
por lo que $f(x,y)\geq 0$ para todo $(x,y)$.
La función no es radial ni convexa globalmente, pero como se verá más adelante,
presenta convexidad local en un entorno del origen.

\section{Existencia y unicidad del mínimo global}

\begin{proposicion}
La función $f$ posee un único mínimo global en el punto $(0,0)$.
\end{proposicion}

\begin{proof}
Para todo $(x,y)\in\mathbb{R}^2$ se cumple
\[
f(x,y)=\frac{\arctan(x^2+y^2)}{1+x^2}\geq 0.
\]
Además,
\[
f(0,0)=\frac{\arctan(0)}{1}=0.
\]
Si $(x,y)\neq(0,0)$, entonces $x^2+y^2>0$ y por tanto $\arctan(x^2+y^2)>0$, lo que implica
$f(x,y)>0$.
De aquí se concluye que $(0,0)$ es el mínimo global y es único.
\end{proof}

\section{Cálculo del gradiente y puntos críticos}

Calculamos las derivadas parciales de $f$:
\[
\frac{\partial f}{\partial x}
=
\frac{
2x(1+x^2)\frac{1}{1+(x^2+y^2)^2}
-
2x\arctan(x^2+y^2)
}{
(1+x^2)^2
},
\]
\[
\frac{\partial f}{\partial y}
=
\frac{
2y
}{
(1+x^2)\left(1+(x^2+y^2)^2\right)
}.
\]

\begin{proposicion}
El único punto crítico de $f$ es el origen $(0,0)$.
\end{proposicion}

\begin{proof}
De la ecuación $\frac{\partial f}{\partial y}=0$ se obtiene directamente $y=0$.
Sustituyendo en la ecuación $\frac{\partial f}{\partial x}=0$ se llega a
\[
x\left(\frac{1+x^2}{1+x^4}-\arctan(x^2)\right)=0.
\]
La expresión entre paréntesis no se anula para $x\neq 0$, por lo que la única solución
es $x=0$.
Por tanto, el único punto crítico es $(0,0)$.
\end{proof}

\section{Análisis de segundo orden}

\begin{proposicion}
La Hessiana de $f$ en el origen es definida positiva.
\end{proposicion}

\begin{proof}
Usando el desarrollo en serie de Taylor de $\arctan(t)$ alrededor de $t=0$,
\[
\arctan(t)=t+O(t^3),
\]
se obtiene, para $(x,y)$ cercano al origen,
\[
f(x,y)=x^2+y^2+O(\|(x,y)\|^4).
\]
De aquí se deduce que
\[
\nabla^2 f(0,0)=
\begin{pmatrix}
2 & 0\\
0 & 2
\end{pmatrix},
\]
la cual es definida positiva.
\end{proof}

Esto garantiza que el origen es un mínimo estricto y que la función es convexa
en un entorno del punto óptimo, aunque no necesariamente de forma global.

\section{Métodos numéricos de optimización}

Se consideran distintos métodos clásicos para la minimización de $f$:
Gradiente Descendente con paso fijo, Gradiente Descendente con paso adaptativo,
el Método de Newton y un esquema híbrido.

La convexidad local alrededor del origen garantiza la convergencia local del
método de Newton, mientras que los métodos de primer orden muestran mayor robustez
ante condiciones iniciales alejadas del óptimo.

\section{Resultados numéricos}

Los experimentos numéricos confirman que todos los métodos convergen al mínimo global
$(0,0)$.
El Método de Newton presenta convergencia cuadrática en un entorno cercano al origen,
mientras que el Gradiente Descendente con paso adaptativo ofrece un buen balance entre
estabilidad y eficiencia global.

\section{Conclusiones}

Se ha realizado un estudio completo de la función
$f(x,y)=\dfrac{\arctan(x^2+y^2)}{1+x^2}$ desde el punto de vista teórico y numérico.
La función posee un mínimo global único en el origen, es suave y presenta convexidad
local suficiente para garantizar la eficacia de métodos de optimización de segundo orden.
Los resultados obtenidos respaldan las conclusiones teóricas y proporcionan una base
sólida para el análisis de algoritmos de optimización en problemas no lineales.

\end{document}